% Options for packages loaded elsewhere
\PassOptionsToPackage{unicode}{hyperref}
\PassOptionsToPackage{hyphens}{url}
%
\documentclass[
]{article}
\usepackage{amsmath,amssymb}
\usepackage{iftex}
\ifPDFTeX
  \usepackage[T1]{fontenc}
  \usepackage[utf8]{inputenc}
  \usepackage{textcomp} % provide euro and other symbols
\else % if luatex or xetex
  \usepackage{unicode-math} % this also loads fontspec
  \defaultfontfeatures{Scale=MatchLowercase}
  \defaultfontfeatures[\rmfamily]{Ligatures=TeX,Scale=1}
\fi
\usepackage{lmodern}
\ifPDFTeX\else
  % xetex/luatex font selection
\fi
% Use upquote if available, for straight quotes in verbatim environments
\IfFileExists{upquote.sty}{\usepackage{upquote}}{}
\IfFileExists{microtype.sty}{% use microtype if available
  \usepackage[]{microtype}
  \UseMicrotypeSet[protrusion]{basicmath} % disable protrusion for tt fonts
}{}
\makeatletter
\@ifundefined{KOMAClassName}{% if non-KOMA class
  \IfFileExists{parskip.sty}{%
    \usepackage{parskip}
  }{% else
    \setlength{\parindent}{0pt}
    \setlength{\parskip}{6pt plus 2pt minus 1pt}}
}{% if KOMA class
  \KOMAoptions{parskip=half}}
\makeatother
\usepackage{xcolor}
\usepackage[margin=1in]{geometry}
\usepackage{color}
\usepackage{fancyvrb}
\newcommand{\VerbBar}{|}
\newcommand{\VERB}{\Verb[commandchars=\\\{\}]}
\DefineVerbatimEnvironment{Highlighting}{Verbatim}{commandchars=\\\{\}}
% Add ',fontsize=\small' for more characters per line
\usepackage{framed}
\definecolor{shadecolor}{RGB}{248,248,248}
\newenvironment{Shaded}{\begin{snugshade}}{\end{snugshade}}
\newcommand{\AlertTok}[1]{\textcolor[rgb]{0.94,0.16,0.16}{#1}}
\newcommand{\AnnotationTok}[1]{\textcolor[rgb]{0.56,0.35,0.01}{\textbf{\textit{#1}}}}
\newcommand{\AttributeTok}[1]{\textcolor[rgb]{0.13,0.29,0.53}{#1}}
\newcommand{\BaseNTok}[1]{\textcolor[rgb]{0.00,0.00,0.81}{#1}}
\newcommand{\BuiltInTok}[1]{#1}
\newcommand{\CharTok}[1]{\textcolor[rgb]{0.31,0.60,0.02}{#1}}
\newcommand{\CommentTok}[1]{\textcolor[rgb]{0.56,0.35,0.01}{\textit{#1}}}
\newcommand{\CommentVarTok}[1]{\textcolor[rgb]{0.56,0.35,0.01}{\textbf{\textit{#1}}}}
\newcommand{\ConstantTok}[1]{\textcolor[rgb]{0.56,0.35,0.01}{#1}}
\newcommand{\ControlFlowTok}[1]{\textcolor[rgb]{0.13,0.29,0.53}{\textbf{#1}}}
\newcommand{\DataTypeTok}[1]{\textcolor[rgb]{0.13,0.29,0.53}{#1}}
\newcommand{\DecValTok}[1]{\textcolor[rgb]{0.00,0.00,0.81}{#1}}
\newcommand{\DocumentationTok}[1]{\textcolor[rgb]{0.56,0.35,0.01}{\textbf{\textit{#1}}}}
\newcommand{\ErrorTok}[1]{\textcolor[rgb]{0.64,0.00,0.00}{\textbf{#1}}}
\newcommand{\ExtensionTok}[1]{#1}
\newcommand{\FloatTok}[1]{\textcolor[rgb]{0.00,0.00,0.81}{#1}}
\newcommand{\FunctionTok}[1]{\textcolor[rgb]{0.13,0.29,0.53}{\textbf{#1}}}
\newcommand{\ImportTok}[1]{#1}
\newcommand{\InformationTok}[1]{\textcolor[rgb]{0.56,0.35,0.01}{\textbf{\textit{#1}}}}
\newcommand{\KeywordTok}[1]{\textcolor[rgb]{0.13,0.29,0.53}{\textbf{#1}}}
\newcommand{\NormalTok}[1]{#1}
\newcommand{\OperatorTok}[1]{\textcolor[rgb]{0.81,0.36,0.00}{\textbf{#1}}}
\newcommand{\OtherTok}[1]{\textcolor[rgb]{0.56,0.35,0.01}{#1}}
\newcommand{\PreprocessorTok}[1]{\textcolor[rgb]{0.56,0.35,0.01}{\textit{#1}}}
\newcommand{\RegionMarkerTok}[1]{#1}
\newcommand{\SpecialCharTok}[1]{\textcolor[rgb]{0.81,0.36,0.00}{\textbf{#1}}}
\newcommand{\SpecialStringTok}[1]{\textcolor[rgb]{0.31,0.60,0.02}{#1}}
\newcommand{\StringTok}[1]{\textcolor[rgb]{0.31,0.60,0.02}{#1}}
\newcommand{\VariableTok}[1]{\textcolor[rgb]{0.00,0.00,0.00}{#1}}
\newcommand{\VerbatimStringTok}[1]{\textcolor[rgb]{0.31,0.60,0.02}{#1}}
\newcommand{\WarningTok}[1]{\textcolor[rgb]{0.56,0.35,0.01}{\textbf{\textit{#1}}}}
\usepackage{graphicx}
\makeatletter
\def\maxwidth{\ifdim\Gin@nat@width>\linewidth\linewidth\else\Gin@nat@width\fi}
\def\maxheight{\ifdim\Gin@nat@height>\textheight\textheight\else\Gin@nat@height\fi}
\makeatother
% Scale images if necessary, so that they will not overflow the page
% margins by default, and it is still possible to overwrite the defaults
% using explicit options in \includegraphics[width, height, ...]{}
\setkeys{Gin}{width=\maxwidth,height=\maxheight,keepaspectratio}
% Set default figure placement to htbp
\makeatletter
\def\fps@figure{htbp}
\makeatother
\setlength{\emergencystretch}{3em} % prevent overfull lines
\providecommand{\tightlist}{%
  \setlength{\itemsep}{0pt}\setlength{\parskip}{0pt}}
\setcounter{secnumdepth}{-\maxdimen} % remove section numbering
\ifLuaTeX
  \usepackage{selnolig}  % disable illegal ligatures
\fi
\IfFileExists{bookmark.sty}{\usepackage{bookmark}}{\usepackage{hyperref}}
\IfFileExists{xurl.sty}{\usepackage{xurl}}{} % add URL line breaks if available
\urlstyle{same}
\hypersetup{
  hidelinks,
  pdfcreator={LaTeX via pandoc}}

\author{}
\date{\vspace{-2.5em}}

\begin{document}

\hypertarget{data-preparation}{%
\section{DATA PREPARATION}\label{data-preparation}}

\hypertarget{primates-neoplasy-project}{%
\subsubsection{PRIMATES' NEOPLASY
PROJECT}\label{primates-neoplasy-project}}

\hypertarget{miguel-ramon-alonso}{%
\subsubsection{MIGUEL RAMON ALONSO}\label{miguel-ramon-alonso}}

\begin{Shaded}
\begin{Highlighting}[]
\NormalTok{knitr}\SpecialCharTok{::}\NormalTok{opts\_knit}\SpecialCharTok{$}\FunctionTok{set}\NormalTok{(}\AttributeTok{root.dir =} \StringTok{"/home/rstudio/NEOPLASY\_PRIMATES"}\NormalTok{)}
\NormalTok{knitr}\SpecialCharTok{::}\NormalTok{opts\_chunk}\SpecialCharTok{$}\FunctionTok{set}\NormalTok{(}\AttributeTok{engine.path =} \FunctionTok{list}\NormalTok{(}
\AttributeTok{python =} \StringTok{\textquotesingle{}/home/rstudio/NEOPLASY\_PRIMATES/TreeCluster/venv/bin/python\textquotesingle{}}\NormalTok{))}
\FunctionTok{getwd}\NormalTok{()}
\end{Highlighting}
\end{Shaded}

\begin{verbatim}
## [1] "/home/rstudio/NEOPLASY_PRIMATES"
\end{verbatim}

\begin{Shaded}
\begin{Highlighting}[]
\FunctionTok{getwd}\NormalTok{()}
\end{Highlighting}
\end{Shaded}

\begin{verbatim}
## [1] "/home/rstudio/NEOPLASY_PRIMATES"
\end{verbatim}

\hypertarget{seed-and-library-loading}{%
\subsubsection{Seed and library
loading}\label{seed-and-library-loading}}

\begin{Shaded}
\begin{Highlighting}[]
\FunctionTok{set.seed}\NormalTok{(}\DecValTok{1998}\NormalTok{)}

\CommentTok{\# General use libraries}
\FunctionTok{library}\NormalTok{(ape)}
\FunctionTok{library}\NormalTok{(RRphylo)}
\FunctionTok{library}\NormalTok{(phytools)}
\FunctionTok{library}\NormalTok{(reticulate)}

\CommentTok{\# Non{-}concordant species retrieval }
\FunctionTok{library}\NormalTok{(stringdist)}
\FunctionTok{library}\NormalTok{(taxize)}
\FunctionTok{library}\NormalTok{(R.utils)}

\CommentTok{\# Sort distances chunk}
\FunctionTok{library}\NormalTok{(adephylo)}
\FunctionTok{library}\NormalTok{(castor)}

\CommentTok{\# Cluster merging}
\FunctionTok{library}\NormalTok{(dplyr)}

\CommentTok{\# Plotting}
\FunctionTok{library}\NormalTok{(ggplot2)}
\FunctionTok{library}\NormalTok{(tidyr)}
\FunctionTok{library}\NormalTok{(dplyr)}
\FunctionTok{library}\NormalTok{(viridis)}
\FunctionTok{library}\NormalTok{(ggtree)}
\FunctionTok{library}\NormalTok{(RColorBrewer)}
\end{Highlighting}
\end{Shaded}

\hypertarget{tree-import}{%
\subsubsection{Tree import}\label{tree-import}}

\begin{Shaded}
\begin{Highlighting}[]
\NormalTok{nw\_tree }\OtherTok{\textless{}{-}} \FunctionTok{read.newick}\NormalTok{(}\StringTok{"Data/233{-}GENOMES/science.abn7829\_data\_s4.nex.tree"}\NormalTok{)}

\CommentTok{\# Give a name to species name attribute}
\NormalTok{tree\_species }\OtherTok{\textless{}{-}}\NormalTok{ nw\_tree}\SpecialCharTok{$}\NormalTok{tip.label}
\end{Highlighting}
\end{Shaded}

\hypertarget{cancer-traitfile-import}{%
\subsubsection{Cancer traitfile import}\label{cancer-traitfile-import}}

\begin{Shaded}
\begin{Highlighting}[]
\NormalTok{cancer\_traits }\OtherTok{\textless{}{-}} \FunctionTok{read.csv}\NormalTok{(}\StringTok{"Data/Neoplasia/species360\_primates\_neoplasia\_20230519.csv"}\NormalTok{, }\AttributeTok{sep =} \StringTok{","}\NormalTok{)}
\end{Highlighting}
\end{Shaded}

\hypertarget{cancer-trait-file-polishing}{%
\subsubsection{Cancer trait file
polishing}\label{cancer-trait-file-polishing}}

\hypertarget{remove-spaces-at-the-end-work-out-name-dissimilarities-and-stuff-like-that}{%
\paragraph{Remove spaces at the end, work out name dissimilarities and
stuff like
that}\label{remove-spaces-at-the-end-work-out-name-dissimilarities-and-stuff-like-that}}

\begin{Shaded}
\begin{Highlighting}[]
\CommentTok{\# Remove any whitespaces at the end of the cells}
\NormalTok{cancer\_traits[] }\OtherTok{\textless{}{-}} \FunctionTok{lapply}\NormalTok{(cancer\_traits, }\ControlFlowTok{function}\NormalTok{(col) }\FunctionTok{trimws}\NormalTok{(col))}

\CommentTok{\# Binarize species name for standarization}
\NormalTok{cancer\_traits}\SpecialCharTok{$}\NormalTok{SPECIES\_BINOMIAL }\OtherTok{\textless{}{-}} \FunctionTok{gsub}\NormalTok{(}\StringTok{" "}\NormalTok{, }\StringTok{"\_"}\NormalTok{, cancer\_traits}\SpecialCharTok{$}\NormalTok{species)}

\CommentTok{\# Need to name the object to work wit it (not sure why)}
\NormalTok{cancer\_species\_bin }\OtherTok{\textless{}{-}}\NormalTok{ cancer\_traits}\SpecialCharTok{$}\NormalTok{SPECIES\_BINOMIAL }
\end{Highlighting}
\end{Shaded}

\hypertarget{dissimilarities-between-the-two-namesets-tree-and-traits}{%
\paragraph{Dissimilarities between the two namesets (tree and
traits)}\label{dissimilarities-between-the-two-namesets-tree-and-traits}}

\begin{Shaded}
\begin{Highlighting}[]
\CommentTok{\# Function to retrieve synonyms for a species from ITIS}
\CommentTok{\# Simplified Function to retrieve accepted name for a species from ITIS}
\NormalTok{get\_synonyms }\OtherTok{\textless{}{-}} \ControlFlowTok{function}\NormalTok{(species\_name) \{}
\NormalTok{  retry }\OtherTok{\textless{}{-}} \ConstantTok{TRUE}
\NormalTok{  retries }\OtherTok{\textless{}{-}} \DecValTok{0}
\NormalTok{  max\_retries }\OtherTok{\textless{}{-}} \DecValTok{15}  \CommentTok{\# Set a maximum number of retries to avoid an infinite loop}
\NormalTok{  syns }\OtherTok{\textless{}{-}} \ConstantTok{NULL}
  
  \ControlFlowTok{while}\NormalTok{ (retry }\SpecialCharTok{\&\&}\NormalTok{ retries }\SpecialCharTok{\textless{}}\NormalTok{ max\_retries) \{}
    \FunctionTok{try}\NormalTok{(}
      \CommentTok{\# Set a 10{-}second timeout for the synonyms query}
      \FunctionTok{withTimeout}\NormalTok{(\{}
\NormalTok{        syns }\OtherTok{\textless{}{-}} \FunctionTok{synonyms}\NormalTok{(species\_name, }\AttributeTok{db =} \StringTok{"itis"}\NormalTok{, }\AttributeTok{rows =} \DecValTok{1}\NormalTok{)}
\NormalTok{      \}, }\AttributeTok{timeout =} \DecValTok{10}\NormalTok{),}
      \AttributeTok{silent =} \ConstantTok{TRUE}
\NormalTok{    )}
    
    \CommentTok{\# Check if the query succeeded or timed out}
    \ControlFlowTok{if}\NormalTok{ (}\SpecialCharTok{!}\FunctionTok{is.null}\NormalTok{(syns) }\SpecialCharTok{\&\&} \FunctionTok{length}\NormalTok{(syns) }\SpecialCharTok{\textgreater{}} \DecValTok{0}\NormalTok{) \{}
\NormalTok{      retry }\OtherTok{\textless{}{-}} \ConstantTok{FALSE}  \CommentTok{\# Query succeeded, so exit the loop}
\NormalTok{    \} }\ControlFlowTok{else}\NormalTok{ \{}
      \FunctionTok{Sys.sleep}\NormalTok{(}\DecValTok{5}\NormalTok{)  }\CommentTok{\# Wait 5 seconds before retrying}
\NormalTok{      retries }\OtherTok{\textless{}{-}}\NormalTok{ retries }\SpecialCharTok{+} \DecValTok{1}
\NormalTok{    \}}
\NormalTok{  \}}
  
  \FunctionTok{return}\NormalTok{(syns)}
\NormalTok{\}}



\CommentTok{\# Identify species that are directly present in the tree}
\NormalTok{direct\_matches }\OtherTok{\textless{}{-}}\NormalTok{ cancer\_species\_bin }\SpecialCharTok{\%in\%}\NormalTok{ tree\_species}
\NormalTok{cancer\_species\_to\_check }\OtherTok{\textless{}{-}}\NormalTok{ cancer\_species\_bin[}\SpecialCharTok{!}\NormalTok{direct\_matches]}

\CommentTok{\# Remove subspecies information from cancer\_species\_to\_check}
\NormalTok{cancer\_species\_to\_check }\OtherTok{\textless{}{-}} \FunctionTok{sapply}\NormalTok{(}\FunctionTok{strsplit}\NormalTok{(cancer\_species\_to\_check, }\StringTok{"\_"}\NormalTok{), }\ControlFlowTok{function}\NormalTok{(x) \{}
  \FunctionTok{paste}\NormalTok{(x[}\DecValTok{1}\SpecialCharTok{:}\DecValTok{2}\NormalTok{], }\AttributeTok{collapse =} \StringTok{"\_"}\NormalTok{)}
\NormalTok{\})}


\CommentTok{\# Check for close string matches}
\NormalTok{dist\_matrix }\OtherTok{\textless{}{-}}\NormalTok{ stringdist}\SpecialCharTok{::}\FunctionTok{stringdistmatrix}\NormalTok{(tree\_species, cancer\_species\_to\_check, }\AttributeTok{method =} \StringTok{"lv"}\NormalTok{)}
\NormalTok{threshold }\OtherTok{\textless{}{-}} \DecValTok{2}
\NormalTok{close\_matches }\OtherTok{\textless{}{-}} \FunctionTok{which}\NormalTok{(dist\_matrix }\SpecialCharTok{\textless{}=}\NormalTok{ threshold, }\AttributeTok{arr.ind =} \ConstantTok{TRUE}\NormalTok{)}
\NormalTok{cancer\_species\_bin[}\FunctionTok{which}\NormalTok{(}\SpecialCharTok{!}\NormalTok{direct\_matches)[close\_matches[, }\DecValTok{2}\NormalTok{]]] }\OtherTok{\textless{}{-}}\NormalTok{ tree\_species[close\_matches[, }\DecValTok{1}\NormalTok{]]}

\CommentTok{\# For the species that aren\textquotesingle{}t direct or close matches, check synonyms}
\NormalTok{remaining\_indices }\OtherTok{\textless{}{-}} \FunctionTok{setdiff}\NormalTok{(}\DecValTok{1}\SpecialCharTok{:}\FunctionTok{length}\NormalTok{(cancer\_species\_to\_check), close\_matches[, }\DecValTok{2}\NormalTok{])}
\ControlFlowTok{for}\NormalTok{ (i }\ControlFlowTok{in}\NormalTok{ remaining\_indices) \{}
\NormalTok{  current\_name }\OtherTok{\textless{}{-}}\NormalTok{ cancer\_species\_to\_check[i]}
  
  \CommentTok{\# Check for synonyms}
\NormalTok{  accepted\_name }\OtherTok{\textless{}{-}} \FunctionTok{get\_synonyms}\NormalTok{(current\_name)}
  
  \ControlFlowTok{if}\NormalTok{ (accepted\_name }\SpecialCharTok{\%in\%}\NormalTok{ tree\_species) \{}
    \FunctionTok{cat}\NormalTok{(}\FunctionTok{paste}\NormalTok{(}\StringTok{"Changing"}\NormalTok{, current\_name, }\StringTok{"to"}\NormalTok{, accepted\_name, }\StringTok{"based on accepted name match.}\SpecialCharTok{\textbackslash{}n}\StringTok{"}\NormalTok{))}
\NormalTok{    cancer\_species\_bin[}\FunctionTok{which}\NormalTok{(}\SpecialCharTok{!}\NormalTok{direct\_matches)[i]] }\OtherTok{\textless{}{-}}\NormalTok{ accepted\_name}
\NormalTok{  \} }\ControlFlowTok{else}\NormalTok{ \{}
    \FunctionTok{cat}\NormalTok{(}\FunctionTok{paste}\NormalTok{(}\StringTok{"Removing"}\NormalTok{, current\_name, }\StringTok{"as it doesn\textquotesingle{}t match synonymically.}\SpecialCharTok{\textbackslash{}n}\StringTok{"}\NormalTok{))}
\NormalTok{    cancer\_species\_bin[}\FunctionTok{which}\NormalTok{(}\SpecialCharTok{!}\NormalTok{direct\_matches)[i]] }\OtherTok{\textless{}{-}} \ConstantTok{NA}
\NormalTok{  \}}
\NormalTok{\}}
\end{Highlighting}
\end{Shaded}

\begin{verbatim}
## ==  1 queries  ===============
\end{verbatim}

\begin{verbatim}
## 
## Retrieving data for taxon 'Ateles_fusciceps'
\end{verbatim}

\begin{verbatim}
## ✔  Found:  Ateles_fusciceps
## ==  Results  =================
## 
## • Total: 1 
## • Found: 1 
## • Not Found: 0
## ==  1 queries  ===============
\end{verbatim}

\begin{verbatim}
## 
## Retrieving data for taxon 'Ateles_fusciceps'
\end{verbatim}

\begin{verbatim}
## ✔  Found:  Ateles_fusciceps
## ==  Results  =================
## 
## • Total: 1 
## • Found: 1 
## • Not Found: 0
## Removing Ateles_fusciceps as it doesn't match synonymically.
## ==  1 queries  ===============
\end{verbatim}

\begin{verbatim}
## 
## Retrieving data for taxon 'Cebus_capucinus'
\end{verbatim}

\begin{verbatim}
## ✔  Found:  Cebus_capucinus
## ==  Results  =================
## 
## • Total: 1 
## • Found: 1 
## • Not Found: 0
## ==  1 queries  ===============
\end{verbatim}

\begin{verbatim}
## 
## Retrieving data for taxon 'Cebus_capucinus'
\end{verbatim}

\begin{verbatim}
## ✔  Found:  Cebus_capucinus
## ==  Results  =================
## 
## • Total: 1 
## • Found: 1 
## • Not Found: 0
## ==  1 queries  ===============
\end{verbatim}

\begin{verbatim}
## 
## Retrieving data for taxon 'Cebus_capucinus'
\end{verbatim}

\begin{verbatim}
## ==  1 queries  ===============
\end{verbatim}

\begin{verbatim}
## 
## Retrieving data for taxon 'Cebus_capucinus'
\end{verbatim}

\begin{verbatim}
## ==  1 queries  ===============
\end{verbatim}

\begin{verbatim}
## 
## Retrieving data for taxon 'Cebus_capucinus'
\end{verbatim}

\begin{verbatim}
## ✔  Found:  Cebus_capucinus
## ==  Results  =================
## 
## • Total: 1 
## • Found: 1 
## • Not Found: 0
## Removing Cebus_capucinus as it doesn't match synonymically.
## ==  1 queries  ===============
\end{verbatim}

\begin{verbatim}
## 
## Retrieving data for taxon 'Macaca_sylvanus'
\end{verbatim}

\begin{verbatim}
## ==  1 queries  ===============
\end{verbatim}

\begin{verbatim}
## 
## Retrieving data for taxon 'Macaca_sylvanus'
\end{verbatim}

\begin{verbatim}
## ✔  Found:  Macaca_sylvanus
## ==  Results  =================
## 
## • Total: 1 
## • Found: 1 
## • Not Found: 0
## Removing Macaca_sylvanus as it doesn't match synonymically.
## ==  1 queries  ===============
\end{verbatim}

\begin{verbatim}
## 
## Retrieving data for taxon 'Nomascus_leucogenys'
\end{verbatim}

\begin{verbatim}
## ✔  Found:  Nomascus_leucogenys
## ==  Results  =================
## 
## • Total: 1 
## • Found: 1 
## • Not Found: 0
## Removing Nomascus_leucogenys as it doesn't match synonymically.
## ==  1 queries  ===============
\end{verbatim}

\begin{verbatim}
## 
## Retrieving data for taxon 'Pithecia_pithecia'
\end{verbatim}

\begin{verbatim}
## ✔  Found:  Pithecia_pithecia
## ==  Results  =================
## 
## • Total: 1 
## • Found: 1 
## • Not Found: 0
## Removing Pithecia_pithecia as it doesn't match synonymically.
## ==  1 queries  ===============
\end{verbatim}

\begin{verbatim}
## 
## Retrieving data for taxon 'Plecturocebus_donacophilus'
\end{verbatim}

\begin{verbatim}
## ✔  Found:  Plecturocebus_donacophilus
## ==  Results  =================
## 
## • Total: 1 
## • Found: 1 
## • Not Found: 0
## Removing Plecturocebus_donacophilus as it doesn't match synonymically.
## ==  1 queries  ===============
\end{verbatim}

\begin{verbatim}
## 
## Retrieving data for taxon 'Saguinus_leucopus'
\end{verbatim}

\begin{verbatim}
## ==  1 queries  ===============
\end{verbatim}

\begin{verbatim}
## 
## Retrieving data for taxon 'Saguinus_leucopus'
\end{verbatim}

\begin{verbatim}
## ✔  Found:  Saguinus_leucopus
## ==  Results  =================
## 
## • Total: 1 
## • Found: 1 
## • Not Found: 0
\end{verbatim}

\begin{verbatim}
## Accepted name(s) is/are 'Oedipomidas leucopus'
\end{verbatim}

\begin{verbatim}
## Using tsn(s) 1207739
\end{verbatim}

\begin{verbatim}
## Removing Saguinus_leucopus as it doesn't match synonymically.
## ==  1 queries  ===============
\end{verbatim}

\begin{verbatim}
## 
## Retrieving data for taxon 'Saimiri_boliviensis'
\end{verbatim}

\begin{verbatim}
## ✔  Found:  Saimiri_boliviensis
## ==  Results  =================
## 
## • Total: 1 
## • Found: 1 
## • Not Found: 0
## ==  1 queries  ===============
\end{verbatim}

\begin{verbatim}
## 
## Retrieving data for taxon 'Saimiri_boliviensis'
\end{verbatim}

\begin{verbatim}
## ✔  Found:  Saimiri_boliviensis
## ==  Results  =================
## 
## • Total: 1 
## • Found: 1 
## • Not Found: 0
## Removing Saimiri_boliviensis as it doesn't match synonymically.
## ==  1 queries  ===============
\end{verbatim}

\begin{verbatim}
## 
## Retrieving data for taxon 'Symphalangus_syndactylus'
\end{verbatim}

\begin{verbatim}
## ✔  Found:  Symphalangus_syndactylus
## ==  Results  =================
## 
## • Total: 1 
## • Found: 1 
## • Not Found: 0
## Removing Symphalangus_syndactylus as it doesn't match synonymically.
\end{verbatim}

\begin{Shaded}
\begin{Highlighting}[]
\CommentTok{\# Note that every time that the above code fails, rest of the execution shuts so as soon as}
\CommentTok{\# query is done, run below code}

\CommentTok{\# Filter out NAs}
\NormalTok{cancer\_species\_bin }\OtherTok{\textless{}{-}}\NormalTok{ cancer\_species\_bin[}\SpecialCharTok{!}\FunctionTok{is.na}\NormalTok{(cancer\_species\_bin)]}
\FunctionTok{names}\NormalTok{(cancer\_species\_bin) }\OtherTok{\textless{}{-}}\NormalTok{ cancer\_species\_bin}
\FunctionTok{print}\NormalTok{(cancer\_species\_bin)}
\end{Highlighting}
\end{Shaded}

\begin{verbatim}
##              Alouatta_caraya           Aotus_griseimembra 
##            "Alouatta_caraya"         "Aotus_griseimembra" 
##             Ateles_geoffroyi            Callimico_goeldii 
##           "Ateles_geoffroyi"          "Callimico_goeldii" 
##         Callithrix_geoffroyi           Callithrix_jacchus 
##       "Callithrix_geoffroyi"         "Callithrix_jacchus" 
##             Cebuella_pygmaea      Cercopithecus_neglectus 
##           "Cebuella_pygmaea"    "Cercopithecus_neglectus" 
##          Cheirogaleus_medius              Colobus_guereza 
##        "Cheirogaleus_medius"            "Colobus_guereza" 
##           Erythrocebus_patas            Eulemur_coronatus 
##         "Erythrocebus_patas"          "Eulemur_coronatus" 
##               Eulemur_macaco               Eulemur_mongoz 
##             "Eulemur_macaco"             "Eulemur_mongoz" 
##                Galago_moholi          Galago_senegalensis 
##              "Galago_moholi"        "Galago_senegalensis" 
##              Gorilla_gorilla                Hylobates_lar 
##            "Gorilla_gorilla"              "Hylobates_lar" 
##                  Lemur_catta      Leontocebus_fuscicollis 
##                "Lemur_catta"    "Leontocebus_fuscicollis" 
##   Leontopithecus_chrysomelas       Leontopithecus_rosalia 
## "Leontopithecus_chrysomelas"     "Leontopithecus_rosalia" 
##          Macaca_fascicularis               Macaca_fuscata 
##        "Macaca_fascicularis"             "Macaca_fuscata" 
##               Macaca_mulatta                 Macaca_nigra 
##             "Macaca_mulatta"               "Macaca_nigra" 
##               Macaca_silenus            Mandrillus_sphinx 
##             "Macaca_silenus"          "Mandrillus_sphinx" 
##              Mico_argentatus           Microcebus_murinus 
##            "Mico_argentatus"         "Microcebus_murinus" 
##             Nasalis_larvatus           Nycticebus_coucang 
##           "Nasalis_larvatus"         "Nycticebus_coucang" 
##          Nycticebus_pygmaeus              Pan_troglodytes 
##        "Nycticebus_pygmaeus"            "Pan_troglodytes" 
##              Papio_hamadryas                 Pongo_abelii 
##            "Papio_hamadryas"               "Pongo_abelii" 
##               Pongo_pygmaeus        Propithecus_coquereli 
##             "Pongo_pygmaeus"      "Propithecus_coquereli" 
##             Saguinus_bicolor           Saguinus_geoffroyi 
##           "Saguinus_bicolor"         "Saguinus_geoffroyi" 
##           Saguinus_imperator               Saguinus_midas 
##         "Saguinus_imperator"             "Saguinus_midas" 
##              Saguinus_mystax             Saguinus_oedipus 
##            "Saguinus_mystax"           "Saguinus_oedipus" 
##             Saimiri_sciureus               Sapajus_apella 
##           "Saimiri_sciureus"             "Sapajus_apella" 
##         Theropithecus_gelada       Trachypithecus_auratus 
##       "Theropithecus_gelada"     "Trachypithecus_auratus" 
##     Trachypithecus_cristatus     Trachypithecus_francoisi 
##   "Trachypithecus_cristatus"   "Trachypithecus_francoisi" 
##      Trachypithecus_obscurus                Varecia_rubra 
##    "Trachypithecus_obscurus"              "Varecia_rubra" 
##            Varecia_variegata 
##          "Varecia_variegata"
\end{verbatim}

\begin{Shaded}
\begin{Highlighting}[]
\DocumentationTok{\#\#\# }\AlertTok{NOTE}\DocumentationTok{ THAT SUBSPECIES WERE REMOVED BECAUSE I WAS BECOMING MAD AND THEY ARE NOT RELEVANT TBF}
\end{Highlighting}
\end{Shaded}

\hypertarget{tree-prunning}{%
\paragraph{Tree prunning}\label{tree-prunning}}

\begin{Shaded}
\begin{Highlighting}[]
\NormalTok{pruned\_tree }\OtherTok{\textless{}{-}} \FunctionTok{treedataMatch}\NormalTok{(}\AttributeTok{tree=}\NormalTok{nw\_tree,}\AttributeTok{y=}\NormalTok{cancer\_species\_bin)}

\CommentTok{\# Rename the variable for posterior ease of use}
\NormalTok{tree }\OtherTok{\textless{}{-}}\NormalTok{ pruned\_tree}\SpecialCharTok{$}\NormalTok{tree}
\FunctionTok{write.tree}\NormalTok{(tree, }\AttributeTok{file=}\StringTok{"Out/tree\_prunning/science.abn7829\_data\_s4.nex.tree.pruned"}\NormalTok{)}

\CommentTok{\# Check how the two trees compare}
\FunctionTok{comparePhylo}\NormalTok{(tree, nw\_tree, }\AttributeTok{plot =} \ConstantTok{TRUE}\NormalTok{, }\AttributeTok{force.rooted =} \ConstantTok{TRUE}\NormalTok{)}
\end{Highlighting}
\end{Shaded}

\includegraphics{data_preparation_files/figure-latex/unnamed-chunk-6-1.pdf}

\begin{verbatim}
## => Comparing tree with nw_tree.
## Trees have different numbers of tips: 53 and 236.
## Tips in nw_tree not in tree : Tupaia_belangeri, Galeopterus_variegatus, Aotus_azarae, Aotus_trivirgatus, Aotus_vociferans, Cebuella_niveiventris, Callibella_humilis, Mico_humeralifer, Mico_spnv, Callithrix_kuhlii, Saguinus_inustus, Saguinus_labiatus, Leontocebus_nigricollis, Leontocebus_illigeri, Cebus_olivaceus, Cebus_unicolor, Cebus_albifrons, Sapajus_macrocephalus, Saimiri_oerstedii, Saimiri_ustus, Saimiri_cassiquiarensis, Saimiri_macrodon, Alouatta_palliata, Alouatta_puruensis, Alouatta_juara, Alouatta_seniculus, Alouatta_nigerrima, Alouatta_macconnelli, Alouatta_belzebul, Alouatta_discolor, Lagothrix_lagothricha, Ateles_paniscus, Ateles_belzebuth, Ateles_marginatus, Ateles_chamek, Cacajao_calvus, Cacajao_melanocephalus, Cacajao_hosomi, Cacajao_ayresi, Chiropotes_albinasus, Chiropotes_sagulatus, Chiropotes_israelita, Pithecia_chrysocephala, Pithecia_albicans, Pithecia_hirsuta, Pithecia_mittermeieri, Pithecia_vanzolinii, Pithecia_pissinattii, Plecturocebus_caligatus, Plecturocebus_brunneus, Plecturocebus_cupreus, Plecturocebus_dubius, Plecturocebus_bernhardi, Plecturocebus_grovesi, Plecturocebus_moloch, Plecturocebus_hoffmansi, Plecturocebus_miltoni, Plecturocebus_cinerascens, Cheracebus_lugens, Cheracebus_lucifer, Cheracebus_torquatus, Cheracebus_regulus, Macaca_assamensis, Macaca_arctoides, Macaca_radiata, Macaca_thibetana, Macaca_cyclopis, Macaca_siberu, Macaca_nemestrina, Macaca_leonina, Macaca_tonkeana, Macaca_maura, Lophocebus_aterrimus, Papio_anubis, Papio_papio, Papio_cynocephalus, Papio_ursinus, Papio_kindae, Mandrillus_leucophaeus, Cercocebus_chrysogaster, Cercocebus_lunulatus, Cercocebus_torquatus, Miopithecus_ogouensis, Cercopithecus_hamlyni, Cercopithecus_nictitans, Cercopithecus_mitis, Cercopithecus_petaurista, Cercopithecus_cephus, Cercopithecus_ascanius, Cercopithecus_pogonias, Cercopithecus_lowei, Cercopithecus_mona, Cercopithecus_diana, Cercopithecus_roloway, Allenopithecus_nigroviridis, Chlorocebus_pygerythrus, Allochrocebus_solatus, Allochrocebus_preussi, Allochrocebus_lhoesti, Trachypithecus_germaini, Trachypithecus_leucocephalus, Trachypithecus_hatinhensis, Trachypithecus_laotum, Trachypithecus_crepusculus, Trachypithecus_phayrei, Trachypithecus_melamera, Trachypithecus_geei, Trachypithecus_pileatus, Semnopithecus_vetulus, Semnopithecus_johnii, Semnopithecus_entellus, Semnopithecus_schistaceus, Semnopithecus_priam, Semnopithecus_hypoleucos, Pygathrix_nigripes, Pygathrix_nemaeus, Pygathrix_cinerea, Rhinopithecus_bieti, Rhinopithecus_roxellana, Presbytis_mitrata, Presbytis_comata, Piliocolobus_badius, Piliocolobus_tephrosceles, Piliocolobus_gordonorum, Piliocolobus_kirkii, Colobus_angolensis, Colobus_polykomos, Homo_sapiens, Pan_paniscus, Gorilla_beringei, Hoolock_hoolock, Hylobates_klossii, Hylobates_agilis, Hylobates_muelleri, Hylobates_abbotti, Nomascus_concolor, Nomascus_siki, Nomascus_gabriellae, Nomascus_annamensis, Tarsius_lariang, Tarsius_wallacei, Cephalopachus_bancanus, Carlito_syrichta, Daubentonia_madagascariensis, Eulemur_rubriventer, Eulemur_albifrons, Eulemur_sanfordi, Eulemur_fulvus, Eulemur_rufus, Eulemur_collaris, Eulemur_flavifrons, Prolemur_simus, Hapalemur_occidentalis, Hapalemur_alaotrensis, Hapalemur_gilberti, Hapalemur_griseus, Hapalemur_meridionalis, Indri_indri, Avahi_laniger, Avahi_peyrierasi, Propithecus_coronatus, Propithecus_tattersalli, Propithecus_verreauxi, Propithecus_edwardsi, Propithecus_diadema, Propithecus_perrieri, Cheirogaleus_major, Mirza_zaza, Lepilemur_ruficaudatus, Lepilemur_dorsalis, Lepilemur_septentrionalis, Lepilemur_ankaranensis, Nycticebus_bengalensis, Loris_lydekkerianus, Loris_tardigradus, Arctocebus_calabarensis, Perodicticus_ibeanus, Perodicticus_potto, Galagoides_demidoff, Otolemur_crassicaudatus, Otolemur_garnettii, Mus_musculus, Oryctolagus_cuniculus.
## Trees have different numbers of nodes: 52 and 235.
## Both trees are rooted.
## Both trees are ultrametric.
## 44 clades in tree not in nw_tree.
## 227 clades in nw_tree not in tree.
## Branching times of clades in common between both trees: see ..$BT
## (node number in parentheses).
## 
## $BT
##                     tree                 nw_tree
## 1 0.728006445700004 (65) 0.728006434700006 (261)
## 2  3.25095038290001 (68)  3.25095037190002 (264)
## 3 0.541204418900008 (69) 0.541204407900011 (265)
## 4  1.52810728990001 (70)  1.52810727890002 (266)
## 5 0.330008522299998 (89) 0.330008511299994 (377)
## 6       1.551334658 (93)       1.551334647 (408)
## 7      1.907136741 (100)  1.90713672999999 (441)
## 8      1.121762388 (105)       1.121762377 (469)
\end{verbatim}

\hypertarget{clustering-by-branch-depth}{%
\subsubsection{Clustering by branch
depth}\label{clustering-by-branch-depth}}

\hypertarget{revised-to-do-the-exploration-node-wise}{%
\subparagraph{Revised to do the exploration
node-wise}\label{revised-to-do-the-exploration-node-wise}}

\begin{Shaded}
\begin{Highlighting}[]
\CommentTok{\# Get distances from all nodes to root}
\NormalTok{sorted\_distances }\OtherTok{\textless{}{-}} \FunctionTok{sort}\NormalTok{(}\FunctionTok{get\_all\_distances\_to\_root}\NormalTok{(tree))}

\CommentTok{\# Identify duplicates rounded to 5 decimal places (for significance)}
\NormalTok{duplicates }\OtherTok{\textless{}{-}} \FunctionTok{duplicated}\NormalTok{(}\FunctionTok{round}\NormalTok{(}\FunctionTok{unlist}\NormalTok{(sorted\_distances), }\DecValTok{5}\NormalTok{))}

\CommentTok{\# Remove duplicates}
\NormalTok{unique\_distances }\OtherTok{\textless{}{-}}\NormalTok{ sorted\_distances[}\SpecialCharTok{!}\NormalTok{duplicates]}

\CommentTok{\# Cut the tree using TreeCluster (bash)}
\ControlFlowTok{for}\NormalTok{ (index }\ControlFlowTok{in} \FunctionTok{seq\_along}\NormalTok{(unique\_distances)) \{}
\NormalTok{  node }\OtherTok{\textless{}{-}}\NormalTok{ unique\_distances[[index]]}
  
  \CommentTok{\# Visualice the node}
\NormalTok{  echo\_command }\OtherTok{\textless{}{-}} \FunctionTok{sprintf}\NormalTok{(}\StringTok{"echo \%f"}\NormalTok{, node)}
  \FunctionTok{system}\NormalTok{(echo\_command)}
    
  \CommentTok{\# Cut the tree by every possible nodal distance and collect the clusters in separated files for further use}
\NormalTok{  output\_path }\OtherTok{\textless{}{-}} \FunctionTok{sprintf}\NormalTok{(}\StringTok{"Out/tree\_clustering/\%d\_cut\_\%f.clusters"}\NormalTok{, index, node)}
\NormalTok{  tree\_cluster\_command }\OtherTok{\textless{}{-}} \FunctionTok{sprintf}\NormalTok{(}\StringTok{"python3 TreeCluster/TreeCluster.py {-}i Out/tree\_prunning/science.abn7829\_data\_s4.nex.tree.pruned {-}t \%f {-}{-}method leaf\_dist\_max {-}v {-}o \%s"}\NormalTok{, node, output\_path)}
  \FunctionTok{system}\NormalTok{(tree\_cluster\_command)}
\NormalTok{\}}
\end{Highlighting}
\end{Shaded}

\hypertarget{now-lets-merge-all-the-clusters-into-a-single-dataframe-including-also-the-families}{%
\subsubsection{Now lets merge all the clusters into a single dataframe,
including also the
families}\label{now-lets-merge-all-the-clusters-into-a-single-dataframe-including-also-the-families}}

\begin{Shaded}
\begin{Highlighting}[]
\CommentTok{\# Identify all cluster files}
\NormalTok{cluster\_files }\OtherTok{\textless{}{-}} \FunctionTok{list.files}\NormalTok{(}\AttributeTok{path =} \StringTok{"Out/tree\_clustering/"}\NormalTok{, }\AttributeTok{full.names =} \ConstantTok{TRUE}\NormalTok{)}

\CommentTok{\# Function to read and process a single cluster file}
\NormalTok{process\_cluster\_file }\OtherTok{\textless{}{-}} \ControlFlowTok{function}\NormalTok{(file\_path) \{}
  \CommentTok{\# Extract just the filename without the extension}
\NormalTok{  filename\_no\_ext }\OtherTok{\textless{}{-}}\NormalTok{ tools}\SpecialCharTok{::}\FunctionTok{file\_path\_sans\_ext}\NormalTok{(}\FunctionTok{basename}\NormalTok{(file\_path))}
  
  \CommentTok{\# Extract the desired portion for the header by splitting on \textquotesingle{}\_\textquotesingle{}}
\NormalTok{  components }\OtherTok{\textless{}{-}} \FunctionTok{strsplit}\NormalTok{(filename\_no\_ext, }\StringTok{"\_"}\NormalTok{)[[}\DecValTok{1}\NormalTok{]]}
\NormalTok{  cut\_point }\OtherTok{\textless{}{-}} \FunctionTok{paste0}\NormalTok{(components[}\FunctionTok{length}\NormalTok{(components) }\SpecialCharTok{{-}} \DecValTok{1}\NormalTok{], }\StringTok{"\_"}\NormalTok{, }\FunctionTok{tail}\NormalTok{(components, }\DecValTok{1}\NormalTok{))}
  
  \CommentTok{\# Read the file into a dataframe}
\NormalTok{  df }\OtherTok{\textless{}{-}} \FunctionTok{read.csv}\NormalTok{(file\_path, }\AttributeTok{sep =} \StringTok{"}\SpecialCharTok{\textbackslash{}t}\StringTok{"}\NormalTok{, }\AttributeTok{stringsAsFactors =} \ConstantTok{FALSE}\NormalTok{)}
  
  \CommentTok{\# Rename columns}
  \FunctionTok{colnames}\NormalTok{(df)[}\FunctionTok{colnames}\NormalTok{(df) }\SpecialCharTok{==} \StringTok{"SequenceName"}\NormalTok{] }\OtherTok{\textless{}{-}} \StringTok{"SPECIES\_BINOMIAL"}
  \FunctionTok{colnames}\NormalTok{(df)[}\FunctionTok{colnames}\NormalTok{(df) }\SpecialCharTok{==} \StringTok{"ClusterNumber"}\NormalTok{] }\OtherTok{\textless{}{-}}\NormalTok{ cut\_point}
  
  \FunctionTok{return}\NormalTok{(df)}
\NormalTok{\}}

\CommentTok{\# Process all cluster files}
\NormalTok{list\_of\_dfs }\OtherTok{\textless{}{-}} \FunctionTok{lapply}\NormalTok{(cluster\_files, process\_cluster\_file)}

\CommentTok{\# Merge all data frames by SequenceName}
\NormalTok{clusters\_df }\OtherTok{\textless{}{-}} \FunctionTok{Reduce}\NormalTok{(}\ControlFlowTok{function}\NormalTok{(x, y) \{}
  \FunctionTok{merge}\NormalTok{(x, y, }\AttributeTok{by =} \StringTok{"SPECIES\_BINOMIAL"}\NormalTok{, }\AttributeTok{all.y =} \ConstantTok{TRUE}\NormalTok{, }\AttributeTok{no.dups =} \ConstantTok{TRUE}\NormalTok{)}
\NormalTok{\}, list\_of\_dfs)}

\CommentTok{\# Keep the SPECIES\_BINOMIAL column fixed and order the rest}
\NormalTok{ordered\_columns }\OtherTok{\textless{}{-}} \FunctionTok{c}\NormalTok{(}\StringTok{"SPECIES\_BINOMIAL"}\NormalTok{, }\FunctionTok{sort}\NormalTok{(}\FunctionTok{colnames}\NormalTok{(clusters\_df)[}\SpecialCharTok{{-}}\DecValTok{1}\NormalTok{]))}

\CommentTok{\# Reorder the columns based on the sorted column names}
\NormalTok{clusters\_df }\OtherTok{\textless{}{-}}\NormalTok{ clusters\_df[, ordered\_columns]}

\CommentTok{\# Remove duplicated cuts and 0.000 cut}
\DocumentationTok{\#\# This removes redundancy, as different, very close cuts have exactly the same clusters}
\NormalTok{clusters\_unique }\OtherTok{\textless{}{-}}\NormalTok{ clusters\_df[}\SpecialCharTok{!}\FunctionTok{duplicated}\NormalTok{(}\FunctionTok{t}\NormalTok{(clusters\_df))]}
\NormalTok{clusters\_unique }\OtherTok{\textless{}{-}}\NormalTok{ clusters\_unique[, }\SpecialCharTok{!}\NormalTok{(}\FunctionTok{names}\NormalTok{(clusters\_unique) }\SpecialCharTok{\%in\%} \StringTok{"cut\_0.000000"}\NormalTok{)]}

\CommentTok{\# Add families from traitfile}
\NormalTok{merged\_traits }\OtherTok{\textless{}{-}} \FunctionTok{merge}\NormalTok{(clusters\_unique, cancer\_traits, }\AttributeTok{by=}\StringTok{"SPECIES\_BINOMIAL"}\NormalTok{, }\AttributeTok{all.x=}\ConstantTok{TRUE}\NormalTok{)}
\NormalTok{clusters\_unique }\OtherTok{\textless{}{-}} \FunctionTok{cbind}\NormalTok{(clusters\_unique[,}\DecValTok{1}\NormalTok{], merged\_traits}\SpecialCharTok{$}\NormalTok{family, clusters\_unique[,}\SpecialCharTok{{-}}\DecValTok{1}\NormalTok{])}

\FunctionTok{colnames}\NormalTok{(clusters\_unique)[}\DecValTok{1}\NormalTok{] }\OtherTok{\textless{}{-}} \StringTok{"SPECIES\_BINOMIAL"}
\FunctionTok{colnames}\NormalTok{(clusters\_unique)[}\DecValTok{2}\NormalTok{] }\OtherTok{\textless{}{-}} \StringTok{"FAMILY"}
\end{Highlighting}
\end{Shaded}

\hypertarget{create-expanded-clusters-file-for-further-analyses}{%
\subsubsection{Create expanded clusters file for further
analyses}\label{create-expanded-clusters-file-for-further-analyses}}

\hypertarget{in-order-to-include-species-which-are-not-present-in-the-phylogeny-but-might-be-interesting-to-see-how-the-phenotype-is-distributed-add-them-to-an-expanded-data-frame-replicating-clustering-data-from-closest-individual}{%
\subparagraph{In order to include species which are not present in the
phylogeny but might be interesting to see how the phenotype is
distributed, add them to an expanded data frame, replicating clustering
data from closest
individual}\label{in-order-to-include-species-which-are-not-present-in-the-phylogeny-but-might-be-interesting-to-see-how-the-phenotype-is-distributed-add-them-to-an-expanded-data-frame-replicating-clustering-data-from-closest-individual}}

\begin{Shaded}
\begin{Highlighting}[]
\NormalTok{clusters\_unique\_expanded }\OtherTok{\textless{}{-}}\NormalTok{ clusters\_unique}

\CommentTok{\# Extract the genus from SPECIES\_BINOMIAL}
\NormalTok{clusters\_unique\_expanded}\SpecialCharTok{$}\NormalTok{GENUS }\OtherTok{\textless{}{-}} \FunctionTok{sapply}\NormalTok{(}\FunctionTok{strsplit}\NormalTok{(}\FunctionTok{as.character}\NormalTok{(clusters\_unique\_expanded}\SpecialCharTok{$}\NormalTok{SPECIES\_BINOMIAL), }\StringTok{"\_"}\NormalTok{), }\StringTok{\textasciigrave{}}\AttributeTok{[}\StringTok{\textasciigrave{}}\NormalTok{, }\DecValTok{1}\NormalTok{)}
\NormalTok{genus\_to\_check }\OtherTok{\textless{}{-}} \FunctionTok{sapply}\NormalTok{(}\FunctionTok{strsplit}\NormalTok{(cancer\_species\_to\_check, }\StringTok{"\_"}\NormalTok{), }\StringTok{\textasciigrave{}}\AttributeTok{[}\StringTok{\textasciigrave{}}\NormalTok{, }\DecValTok{1}\NormalTok{)}

\CommentTok{\# Loop through cancer\_species\_to\_check and match with merged\_traits}
\ControlFlowTok{for}\NormalTok{(species }\ControlFlowTok{in}\NormalTok{ cancer\_species\_to\_check)\{}
  
  \CommentTok{\# Find the genus of the species}
\NormalTok{  genus }\OtherTok{\textless{}{-}} \FunctionTok{strsplit}\NormalTok{(species, }\StringTok{"\_"}\NormalTok{)[[}\DecValTok{1}\NormalTok{]][}\DecValTok{1}\NormalTok{]}
  
  \CommentTok{\# Check if the genus exists in merged\_traits}
\NormalTok{  matching\_index }\OtherTok{\textless{}{-}} \FunctionTok{which}\NormalTok{(clusters\_unique\_expanded}\SpecialCharTok{$}\NormalTok{GENUS }\SpecialCharTok{==}\NormalTok{ genus)}
  
  \CommentTok{\# If a match is found, duplicate the row and modify the SPECIES\_BINOMIAL}
  \ControlFlowTok{if}\NormalTok{(}\FunctionTok{length}\NormalTok{(matching\_index) }\SpecialCharTok{\textgreater{}} \DecValTok{0}\NormalTok{)\{}
\NormalTok{    new\_row }\OtherTok{\textless{}{-}}\NormalTok{ clusters\_unique\_expanded[matching\_index[}\DecValTok{1}\NormalTok{],]}
\NormalTok{    new\_row}\SpecialCharTok{$}\NormalTok{SPECIES\_BINOMIAL }\OtherTok{\textless{}{-}}\NormalTok{ species}
\NormalTok{    clusters\_unique\_expanded }\OtherTok{\textless{}{-}} \FunctionTok{rbind}\NormalTok{(clusters\_unique\_expanded, new\_row)}
\NormalTok{  \}}
\NormalTok{\}}

\CommentTok{\# Remove genus column}
\NormalTok{clusters\_unique\_expanded}\SpecialCharTok{$}\NormalTok{GENUS }\OtherTok{\textless{}{-}} \ConstantTok{NULL}
\end{Highlighting}
\end{Shaded}

\hypertarget{visualicing-cluster-distribution-with-species-present-in-the-phylogeny-and-trait-values}{%
\subsubsection{Visualicing cluster distribution with species present in
the phylogeny and trait
values}\label{visualicing-cluster-distribution-with-species-present-in-the-phylogeny-and-trait-values}}

\begin{Shaded}
\begin{Highlighting}[]
\CommentTok{\# Extracting tip labels order from the phylo object}
\NormalTok{ordered\_species }\OtherTok{\textless{}{-}}\NormalTok{ tree}\SpecialCharTok{$}\NormalTok{tip.label}

\CommentTok{\# Create a new factor with the desired order for SPECIES\_BINOMIAL}
\NormalTok{clusters\_unique}\SpecialCharTok{$}\NormalTok{SPECIES\_BINOMIAL }\OtherTok{\textless{}{-}} \FunctionTok{factor}\NormalTok{(clusters\_unique}\SpecialCharTok{$}\NormalTok{SPECIES\_BINOMIAL, }\AttributeTok{levels =}\NormalTok{ ordered\_species)}

\CommentTok{\# Convert the data frame from wide to long format}
\NormalTok{clusters\_long }\OtherTok{\textless{}{-}}\NormalTok{ clusters\_unique }\SpecialCharTok{\%\textgreater{}\%}
  \FunctionTok{gather}\NormalTok{(}\AttributeTok{key =} \StringTok{"cut"}\NormalTok{, }\AttributeTok{value =} \StringTok{"cluster"}\NormalTok{, }\FunctionTok{starts\_with}\NormalTok{(}\StringTok{"cut"}\NormalTok{))}

\CommentTok{\# Add a cut for FAMILY to visualize it alongside the other cuts}
\NormalTok{clusters\_long }\OtherTok{\textless{}{-}} \FunctionTok{rbind}\NormalTok{(clusters\_long, }\FunctionTok{data.frame}\NormalTok{(}\AttributeTok{SPECIES\_BINOMIAL =}\NormalTok{ clusters\_unique}\SpecialCharTok{$}\NormalTok{SPECIES\_BINOMIAL, }\AttributeTok{cut =} \StringTok{"FAMILY"}\NormalTok{, }\AttributeTok{cluster =} \FunctionTok{as.character}\NormalTok{(clusters\_unique}\SpecialCharTok{$}\NormalTok{FAMILY), }\AttributeTok{FAMILY =}\NormalTok{ clusters\_unique}\SpecialCharTok{$}\NormalTok{FAMILY))}


\CommentTok{\# Calculate the number of unique cluster values and families}
\NormalTok{num\_clusters }\OtherTok{\textless{}{-}} \FunctionTok{length}\NormalTok{(}\FunctionTok{unique}\NormalTok{(clusters\_long}\SpecialCharTok{$}\NormalTok{cluster))}
\NormalTok{num\_families }\OtherTok{\textless{}{-}} \FunctionTok{length}\NormalTok{(}\FunctionTok{unique}\NormalTok{(clusters\_unique}\SpecialCharTok{$}\NormalTok{FAMILY))}

\CommentTok{\# Colors for clusters using the Dark2 palette for distinct colors}
\NormalTok{cluster\_colors }\OtherTok{\textless{}{-}}\NormalTok{ RColorBrewer}\SpecialCharTok{::}\FunctionTok{brewer.pal}\NormalTok{(}\FunctionTok{min}\NormalTok{(num\_clusters, }\DecValTok{8}\NormalTok{), }\StringTok{"Dark2"}\NormalTok{)}

\CommentTok{\# If you have more than 8 clusters, recycle the colors}
\ControlFlowTok{if}\NormalTok{(num\_clusters }\SpecialCharTok{\textgreater{}} \DecValTok{8}\NormalTok{)\{}
\NormalTok{  cluster\_colors }\OtherTok{\textless{}{-}} \FunctionTok{rep}\NormalTok{(cluster\_colors, }\FunctionTok{ceiling}\NormalTok{(num\_clusters}\SpecialCharTok{/}\DecValTok{8}\NormalTok{))[}\DecValTok{1}\SpecialCharTok{:}\NormalTok{num\_clusters]}
\NormalTok{\}}

\CommentTok{\# Colors for families using the Dark2 palette for distinct colors}
\NormalTok{family\_colors }\OtherTok{\textless{}{-}}\NormalTok{ RColorBrewer}\SpecialCharTok{::}\FunctionTok{brewer.pal}\NormalTok{(}\FunctionTok{min}\NormalTok{(num\_families, }\DecValTok{8}\NormalTok{), }\StringTok{"Dark2"}\NormalTok{)}

\CommentTok{\# If you have more than 8 families, recycle the colors}
\ControlFlowTok{if}\NormalTok{(num\_families }\SpecialCharTok{\textgreater{}} \DecValTok{8}\NormalTok{)\{}
\NormalTok{  family\_colors }\OtherTok{\textless{}{-}} \FunctionTok{rep}\NormalTok{(family\_colors, }\FunctionTok{ceiling}\NormalTok{(num\_families}\SpecialCharTok{/}\DecValTok{8}\NormalTok{))[}\DecValTok{1}\SpecialCharTok{:}\NormalTok{num\_families]}
\NormalTok{\}}


\CommentTok{\# Combine the colors and levels}
\NormalTok{all\_colors }\OtherTok{\textless{}{-}} \FunctionTok{c}\NormalTok{(cluster\_colors, family\_colors)}
\NormalTok{all\_levels }\OtherTok{\textless{}{-}} \FunctionTok{unique}\NormalTok{(}\FunctionTok{c}\NormalTok{(}\FunctionTok{as.character}\NormalTok{(clusters\_long}\SpecialCharTok{$}\NormalTok{cluster), }\FunctionTok{as.character}\NormalTok{(clusters\_unique}\SpecialCharTok{$}\NormalTok{FAMILY)))}

\CommentTok{\# Plot}
\FunctionTok{ggplot}\NormalTok{(clusters\_long, }\FunctionTok{aes}\NormalTok{(}\AttributeTok{x =}\NormalTok{ cut, }\AttributeTok{y =}\NormalTok{ SPECIES\_BINOMIAL)) }\SpecialCharTok{+}
  \FunctionTok{geom\_tile}\NormalTok{(}\FunctionTok{aes}\NormalTok{(}\AttributeTok{fill =}\NormalTok{ cluster), }\AttributeTok{color =} \StringTok{"white"}\NormalTok{, }\AttributeTok{linewidth =} \FloatTok{0.5}\NormalTok{) }\SpecialCharTok{+}
  \FunctionTok{scale\_fill\_manual}\NormalTok{(}\AttributeTok{values =} \FunctionTok{setNames}\NormalTok{(all\_colors, all\_levels)) }\SpecialCharTok{+}
  \FunctionTok{scale\_x\_discrete}\NormalTok{(}\AttributeTok{expand =} \FunctionTok{c}\NormalTok{(}\DecValTok{0}\NormalTok{,}\DecValTok{0}\NormalTok{)) }\SpecialCharTok{+}
  \FunctionTok{labs}\NormalTok{(}\AttributeTok{y =} \StringTok{"Species"}\NormalTok{, }\AttributeTok{x =} \StringTok{"Cut \& Family"}\NormalTok{, }\AttributeTok{fill =} \StringTok{"Value/Group"}\NormalTok{) }\SpecialCharTok{+}
  \FunctionTok{theme}\NormalTok{(}\AttributeTok{axis.text.x =} \FunctionTok{element\_text}\NormalTok{(}\AttributeTok{angle =} \DecValTok{90}\NormalTok{, }\AttributeTok{hjust =} \DecValTok{1}\NormalTok{),}
        \AttributeTok{panel.spacing.x =} \FunctionTok{unit}\NormalTok{(}\DecValTok{0}\NormalTok{, }\StringTok{"lines"}\NormalTok{)) }\SpecialCharTok{+}
  \FunctionTok{theme\_minimal}\NormalTok{()}
\end{Highlighting}
\end{Shaded}

\includegraphics{data_preparation_files/figure-latex/unnamed-chunk-10-1.pdf}

\hypertarget{visualicing-cluster-distribution-with-species-not-present-in-the-phylogeny}{%
\subsubsection{Visualicing cluster distribution with species not present
in the
phylogeny}\label{visualicing-cluster-distribution-with-species-not-present-in-the-phylogeny}}

\begin{Shaded}
\begin{Highlighting}[]
\CommentTok{\# Factorice and order by family}
\NormalTok{clusters\_unique\_expanded}\SpecialCharTok{$}\NormalTok{FAMILY }\OtherTok{\textless{}{-}} \FunctionTok{as.factor}\NormalTok{(}\FunctionTok{sort}\NormalTok{(clusters\_unique\_expanded}\SpecialCharTok{$}\NormalTok{FAMILY))}

\CommentTok{\# Convert the data frame from wide to long format}
\NormalTok{clusters\_expanded\_long }\OtherTok{\textless{}{-}}\NormalTok{ clusters\_unique\_expanded }\SpecialCharTok{\%\textgreater{}\%}
  \FunctionTok{gather}\NormalTok{(}\AttributeTok{key =} \StringTok{"cut"}\NormalTok{, }\AttributeTok{value =} \StringTok{"cluster"}\NormalTok{, }\FunctionTok{starts\_with}\NormalTok{(}\StringTok{"cut"}\NormalTok{))}

\CommentTok{\# Add a cut for FAMILY to visualize it alongside the other cuts}
\NormalTok{clusters\_expanded\_long }\OtherTok{\textless{}{-}} \FunctionTok{rbind}\NormalTok{(clusters\_expanded\_long, }\FunctionTok{data.frame}\NormalTok{(}\AttributeTok{SPECIES\_BINOMIAL =}\NormalTok{ clusters\_unique\_expanded}\SpecialCharTok{$}\NormalTok{SPECIES\_BINOMIAL, }\AttributeTok{cut =} \StringTok{"FAMILY"}\NormalTok{, }\AttributeTok{cluster =} \FunctionTok{as.character}\NormalTok{(clusters\_unique\_expanded}\SpecialCharTok{$}\NormalTok{FAMILY), }\AttributeTok{FAMILY =}\NormalTok{ clusters\_unique\_expanded}\SpecialCharTok{$}\NormalTok{FAMILY))}


\CommentTok{\# Calculate the number of unique cluster values and families}
\NormalTok{num\_clusters }\OtherTok{\textless{}{-}} \FunctionTok{length}\NormalTok{(}\FunctionTok{unique}\NormalTok{(clusters\_expanded\_long}\SpecialCharTok{$}\NormalTok{cluster))}
\NormalTok{num\_families }\OtherTok{\textless{}{-}} \FunctionTok{length}\NormalTok{(}\FunctionTok{unique}\NormalTok{(clusters\_unique\_expanded}\SpecialCharTok{$}\NormalTok{FAMILY))}


\CommentTok{\# Colors for clusters using the Dark2 palette for distinct colors}
\NormalTok{cluster\_colors }\OtherTok{\textless{}{-}}\NormalTok{ RColorBrewer}\SpecialCharTok{::}\FunctionTok{brewer.pal}\NormalTok{(}\FunctionTok{min}\NormalTok{(num\_clusters, }\DecValTok{8}\NormalTok{), }\StringTok{"Dark2"}\NormalTok{)}

\CommentTok{\# If you have more than 8 clusters, recycle the colors}
\ControlFlowTok{if}\NormalTok{(num\_clusters }\SpecialCharTok{\textgreater{}} \DecValTok{8}\NormalTok{)\{}
\NormalTok{  cluster\_colors }\OtherTok{\textless{}{-}} \FunctionTok{rep}\NormalTok{(cluster\_colors, }\FunctionTok{ceiling}\NormalTok{(num\_clusters}\SpecialCharTok{/}\DecValTok{8}\NormalTok{))[}\DecValTok{1}\SpecialCharTok{:}\NormalTok{num\_clusters]}
\NormalTok{\}}

\CommentTok{\# Colors for families using the Dark2 palette for distinct colors}
\NormalTok{family\_colors }\OtherTok{\textless{}{-}}\NormalTok{ RColorBrewer}\SpecialCharTok{::}\FunctionTok{brewer.pal}\NormalTok{(}\FunctionTok{min}\NormalTok{(num\_families, }\DecValTok{8}\NormalTok{), }\StringTok{"Dark2"}\NormalTok{)}

\CommentTok{\# If you have more than 8 families, recycle the colors}
\ControlFlowTok{if}\NormalTok{(num\_families }\SpecialCharTok{\textgreater{}} \DecValTok{8}\NormalTok{)\{}
\NormalTok{  family\_colors }\OtherTok{\textless{}{-}} \FunctionTok{rep}\NormalTok{(family\_colors, }\FunctionTok{ceiling}\NormalTok{(num\_families}\SpecialCharTok{/}\DecValTok{8}\NormalTok{))[}\DecValTok{1}\SpecialCharTok{:}\NormalTok{num\_families]}
\NormalTok{\}}

\CommentTok{\# Combine the colors and levels}
\NormalTok{all\_colors }\OtherTok{\textless{}{-}} \FunctionTok{c}\NormalTok{(cluster\_colors, family\_colors)}
\NormalTok{all\_levels }\OtherTok{\textless{}{-}} \FunctionTok{unique}\NormalTok{(}\FunctionTok{c}\NormalTok{(}\FunctionTok{as.character}\NormalTok{(clusters\_expanded\_long}\SpecialCharTok{$}\NormalTok{cluster), }\FunctionTok{as.character}\NormalTok{(clusters\_unique\_expanded}\SpecialCharTok{$}\NormalTok{FAMILY)))}

\CommentTok{\# Plot}
\FunctionTok{ggplot}\NormalTok{(clusters\_expanded\_long, }\FunctionTok{aes}\NormalTok{(}\AttributeTok{x =}\NormalTok{ cut, }\AttributeTok{y =}\NormalTok{ SPECIES\_BINOMIAL)) }\SpecialCharTok{+}
  \FunctionTok{geom\_tile}\NormalTok{(}\FunctionTok{aes}\NormalTok{(}\AttributeTok{fill =}\NormalTok{ cluster), }\AttributeTok{color =} \StringTok{"white"}\NormalTok{, }\AttributeTok{linewidth =} \FloatTok{0.5}\NormalTok{) }\SpecialCharTok{+}
  \FunctionTok{scale\_fill\_manual}\NormalTok{(}\AttributeTok{values =} \FunctionTok{setNames}\NormalTok{(all\_colors, all\_levels)) }\SpecialCharTok{+}
  \FunctionTok{scale\_x\_discrete}\NormalTok{(}\AttributeTok{expand =} \FunctionTok{c}\NormalTok{(}\DecValTok{0}\NormalTok{,}\DecValTok{0}\NormalTok{)) }\SpecialCharTok{+}
  \FunctionTok{labs}\NormalTok{(}\AttributeTok{y =} \StringTok{"Species"}\NormalTok{, }\AttributeTok{x =} \StringTok{"Cut \& Family"}\NormalTok{, }\AttributeTok{fill =} \StringTok{"Value/Group"}\NormalTok{) }\SpecialCharTok{+}
  \FunctionTok{theme}\NormalTok{(}\AttributeTok{axis.text.x =} \FunctionTok{element\_text}\NormalTok{(}\AttributeTok{angle =} \DecValTok{90}\NormalTok{, }\AttributeTok{hjust =} \DecValTok{1}\NormalTok{),}
        \AttributeTok{panel.spacing.x =} \FunctionTok{unit}\NormalTok{(}\DecValTok{0}\NormalTok{, }\StringTok{"lines"}\NormalTok{)) }\SpecialCharTok{+}
  \FunctionTok{theme\_minimal}\NormalTok{()}
\end{Highlighting}
\end{Shaded}

\includegraphics{data_preparation_files/figure-latex/unnamed-chunk-11-1.pdf}

\end{document}
